\documentclass[11pt]{article}
\usepackage[a4paper, hmargin={2cm, 2.5cm},
vmargin={2.5cm, 2.5cm}]{geometry}
\usepackage[utf8]{inputenc}
\usepackage{amsmath,amsfonts,amssymb,wasysym}

\title{Glossary}
\author{Thea Quistgaard}
\date{Master thesis 2020/2021}

\begin{document}
\maketitle

\section{Glossary}
\begin{itemize}
	\item \textbf{Accumulation (rate)} Mass addition of snow/firn to  ice-sheet surface. Equal to the snowfall rate minus rate of loss due to wind, sublimination and melt.
	
	\item \textbf{Advection} Transport of a substance by bulk motion, along with the bulk flux. Bulk motion is the total considered motion of a collection of particles. Advection is driven by the gradient of mechanical energy in the substance/solution. Advection in firn is due to gravitational energy and ice flow influence.	
	
	\item \textbf{Basal sliding} The effect of a glacier moving down slope as a bulk, due to water layer at bedrock created by pressure of overhead weight.
	
	\item \textbf{(Pore) Close-off} The depth/age/density at which all bubbles/pores in firn is sealed off. The bottom of the lock-in zone in firn.
	
	\item \textbf{Connectivity}
	
	\item \textbf{Densification (rate)} The proportional change in air space between (firn)particles
	
	\item \textbf{Diffusion} Molecular exchange driven by concentration gradients in a matter column.
	
	\item \textbf{Diffusivity} Describes the proportinality between the molecular flux due to molecular diffusion and the concentration gradient driving force in a mixed substance.
	
	\item \textbf{Firn} Leftover snow, not melted, from past seasons, recrystallized to a denser substance than that of newly fallen snow. Ice in an intermediate stage between snow and glacial ice.
	
	\item \textbf{Fractionation} Fractionation in the water annual cycle is a process of separation of water isotopes due to a phase transition. When water vapor condensates, the heavy isotopes will have a slight tendency to condensate, and the water droplets will consist of a higher concentration of the heavier isotopes. During cold conditions, air masses over Greenland will have been cooled down more on their way, forming more precipitation, which holds the heavier isotopes. Thus the precipitation falling over Greenland will then be more depleted from the heavier isotopes, showing fractionation thereby, and showing lower $\delta^{18} \text{O}$ values.
	
	\item \textbf{Holocene} The name given to the last ~12000 years of Earth's history which is the time since the end of the last major glacial epoch. A generally warm period.
	
	\item \textbf{Lock-in (zone)} A depth range in firn where a process of enclosure of gas in bubbles or pores takes place.
	
	\item \textbf{Nyquist Frequency} Half of the sampling rate, $\Delta$, of a discretely measured signal. Used when computing PSDs.
	
	\item \textbf{Percolation} The movement and filtering of a substance through porous materials. Considering glaciers, the issue is water percolating through firn or ice, often as melt water.	
	
	\item \textbf{Plastic flow} Deformation of a material otherwise rigid, due to enough severe stress, making the material approximately behave like a Newtonian liquid. Affects ice buried underneath more than about 50 meters. Central and upper parts of glacier moves faster than sides and bottom as friction against bedrock slows down flow.
	
	\item \textbf{Porosity} A measure of the void spaces between particles in a material. Given as a fraction of volume of void over total volume.
	
	\item \textbf{Proxy} A physical measurable variable that indicates another variables previous state. Isotopes as a proxy for temperature. 
	
	\item \textbf{Spatial isotope slope} The slope of the linear relation between water isotopes and temperature.
	
	\item \textbf{Strain} A measure of how much an object is stretched or deformed. Often dealing with the change of length of object.
	
	\item \textbf{Strain, longitudinal} Measure of the amount of stretching or compressing due to (restoring) forces(stresses). Combined, longitudinal stress.
	
	\item \textbf{Stress} The restoring force (per unit area) acting from a body do to a subjected deforming force. Equal in magnitude but opposite.
\end{itemize}

\section{Physics Glossary}
\begin{itemize}
	\item \textbf{Arrhenius plot}
	
	\item \textbf{Arrhenius type rate constant}
	
	\item \textbf{Autoregressive process (AR-)} 
	
	\item \textbf{(De)Convolution}
	
	\item \textbf{Diffusion length} Mean displacement of a water molecule along the z-axis of a firn column.

	\item \textbf{Fick's laws}

	\item \textbf{(Gaussian) Filter}

	\item \textbf{Herron-Langway densification model}

	\item \textbf{Power Spectral Density}

	\item \textbf{Signal to Noise Ratio}

	\item \textbf{Sorge's law}

	\item \textbf{Thinning function}

	\item \textbf{Transfer function}
\end{itemize}

\section{Acronyms}
\begin{itemize}
	\item \textbf{CFM} Community Firn Model

	\item \textbf{PSD} Power Spectral Density

	\item \textbf{WIS} Water Isotopic Signal

	\item \textbf{SNR} Signal to Noise Ratio
	
	\item \textbf{ECM} Electrical Conductivity Measurements
	
	\item \textbf{DEP} DiElectric Profiling 
\end{itemize}
\end{document}