%!TeX root = Chapter_Introduction
\documentclass[../../CompleteThesis/Complete_1stDraft]{subfiles}

\begin{document}
This is an amazing introduction that shows that I know all the stuff that I have done in my thesis. I am clever clever clever.

\section[Laki and Tambora]{Laki and Tambora in Recorded History}
\todo{INTRO: References missing in entire section}

In the not-so-distant past two volcanic horizons have been of great importance for this thesis, namely the Laki eruption in 1783 and the Tambora eruption of 1815. Interestingly, these eruptions have not only impacted the geophysical world, but has left their footprints on the history of Man in politics[REFERENCE], sociology[REFERENCE], arts[REFERENCE] and philosophy[REFERENCE]. On the eighth day of June in 1783 a volcanic fissure located in the southern part of Iceland was central for a global climatic change. The fissure Lakagígar or more commonly known as Laki refering to the central mountain, erupted with violent phreagomagmatic explosions due to the basalt magma being exposed to ground water. The eruption was given a Volcanic Explosivity Index(VEI) of 4, corresponding to the magnitude of the much later 2010 Eyjafjallajökull Icelandic eruption. For the next eight months the fissure continued to emit great amounts of sulfuric aerosols into the atmosphere, resulting locally in Iceland in catastrophic mass famine, due to loss of livestock to poisoning, with up to 25 \% of the population dying from starvation and poisoning from the volcanic gasses. Globally, the eruption caused a huge amount of sulfur dioxide to be spewed into the northern hemisphere which led to a global drop in temperatures and a generally more extreme climate. In the European regions the following summer was much warmer than usual with many thunderstorms to follow. The winter of 1783-84 was subsequently extreme, with long periods of continuous frost. In France the late 1780's were marked by several years with droughts in the summer and frost in the winter, which contributed greatly to a rise in poverty and famine, and creating a greater division between the people and the rulers. Along with a growing dismay and distrust in the ruling forces the climatic changes due to Laki and a number of other climatic disruptions the French political situation finally climaxed in the French revolution of 1789. [REFERENCES!!!!]\\
32 years later on April 5 in 1815 an even more powerful eruption ensued: the eruption of Mount Tambora on the, now, Indonesian island Sumbawa. This eruption had a Volcanic Explosivity Index of 7, which makes it the most powerful in the recorded history of humankind. Considering that the VEI is defined as a logarithmic scale - at least for indices larger than VEI-3 - the Tambora eruption, though located just south of the Equator, impacted the entire globe as well as the European continent in at least the same magnitude as the 32 years prior Laki eruption. Locally, it was estimated to cause at least 10,000 direct deaths and many tens of thousands more due to sulfur dioxid poisoning, famine and disease. In many contexts the year of 1816 following the event became known as "\textit{The Year Without a Summer}", as the ashes from the eruption column dispersed across the world and lowered global temperatures. This significant climate change though was not just a consequence of the Tambora eruption, but was pushed by a number of climatic forcings, some due to several previous volcanic activities around the globe. Combined, these effects coincided in a drop in global temperature by about 0.4 to 0.7 $^{\circ}$C. This climatic change affected the entire globe by disrupting the Indian monsoons, causing a number of failed harvests, laying ground to severe typhus epidemics in southeast Europe and destroying crops and causing potato, oat and what harvest failure, especially in Ireland. Since the eruption had so severe consequences for the day to day lives of many people, the aftermath all around the world has been one of the greatest documented in recorded time, with a clear impact on the works of many artists, among them Lord Byron and J. M. W Turner[REFERENCES!!!!]. Although both eruptions caused many a tragedy, there is beauty in using these events as volcanic horizons in ice core records. Given the severity of both eruptions, they have been so well documented in historical records that they make up solid and certain pillars in developing a temporal map of the past life of our ever so active earth. For that and for their brutal beauty they will remain in human history for a great while to come.

\section[A Rare Gem]{A Rare Gem of Knowledge}
\todo{INTRO: This entire section needs to be written}
\section[Paleotemperature and -climate]{The Importance of Mapping Paleotemperature and -climate}
\todo{INTRO: This entire section needs to be written}
\end{document}