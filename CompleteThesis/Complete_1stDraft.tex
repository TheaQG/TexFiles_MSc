\documentclass[11pt]{memoir}
\usepackage[utf8]{inputenc}
\usepackage{amsmath,amsfonts,amssymb,wasysym}
\usepackage{media9, graphicx, wrapfig}
\usepackage{caption}
\usepackage{graphics,epsfig}
\usepackage{tabularx}	
\usepackage{subcaption}
\usepackage{mathrsfs}
\usepackage{listings}
\usepackage{xcolor}
\usepackage{pgf}
\usepackage{appendix}
\usepackage{etoolbox}
\usepackage{booktabs}
\usepackage{gensymb}
\usepackage{verse}
\usepackage{adjustbox}
\usepackage{array}	 
\usepackage{todonotes}
\usepackage{xcolor}

\definecolor{codegreen}{rgb}{0,0.6,0}
\definecolor{codegray}{rgb}{0.5,0.5,0.5}
\definecolor{codepurple}{rgb}{0.58,0,0.82}
\definecolor{backcolour}{rgb}{0.95,0.95,0.92}

\lstdefinestyle{mystyle}{
	backgroundcolor=\color{backcolour},   
	commentstyle=\color{codegreen},
	keywordstyle=\color{magenta},
	numberstyle=\tiny\color{codegray},
	stringstyle=\color{codepurple},
	basicstyle=\ttfamily\footnotesize,
	breakatwhitespace=false,         
	breaklines=true,                 
	captionpos=b,                    
	keepspaces=true,                 
	numbers=left,                    
	numbersep=5pt,                  
	showspaces=false,                
	showstringspaces=false,
	showtabs=false,                  
	tabsize=2
}

\lstset{style=mystyle}
%\lstset{inputpath=../CodeForListings/}


\usepackage{tikz}
\usetikzlibrary{shapes.geometric, arrows}
\tikzstyle{startstop} = [rectangle, rounded corners, minimum width=3cm, minimum height=1cm,text centered, draw=black, fill=red!30]
\tikzstyle{io} = [trapezium, trapezium left angle=75, trapezium right angle=105, minimum width=2cm, minimum height=1cm, text centered, draw=black, fill=blue!30]
\tikzstyle{process} = [rectangle, minimum width=3cm, minimum height=1cm, text centered, draw=black, fill=orange!30]
\tikzstyle{decision} = [rectangle,, rounded corners, minimum width=3.5cm, minimum height=1.5cm, text centered, draw=black, fill=green!30]
\tikzstyle{arrow} = [thick,->,>=stealth]


\newcommand{\attrib}[1]{%
\nopagebreak{\raggedleft\footnotesize #1\par}}
\renewcommand{\poemtitlefont}{\normalfont\large\itshape\centering}
		

\captionsetup[sub]{font=footnotesize,labelfont={bf,sf}}
\setsecnumdepth{subsection}
\settocdepth{subsubsection}
\graphicspath{{../../Figures/}{../Figures/}}



\BeforeBeginEnvironment{appendices}{\clearpage}

\usepackage{subfiles}





\title{\textbf{Laki To Tambora} \\ A study of ice cores}
\author{Thea Quistgaard}
\date{Master Thesis 2020/2021}

\begin{document}
\lstset{language=python}
\frontmatter

\maketitle

\begin{figure}[h]
	\centering
	\includegraphics[width=0.8\textwidth]{TurnerEruptionofSoufriere-Mountains.jpg}
	\caption{J. M. W. Turner: \textit{The Eruption of the Soufriere Mountains}}
	\label{Fig:Turner}
\end{figure}


\newpage


\poemtitle{Darkness}
\settowidth{\versewidth}{Rayless, and pathless, and the icy earth}
\begin{verse}[\versewidth]
I had a dream, which was not all a dream.\\
The bright sun was extinguish'd, and the stars\\
Did wander darkling in the eternal space,\\
Rayless, and pathless, and the icy earth\\
Swung blind and blackening in the moonless air;\\
Morn came and went—and came, and brought no day.
\end{verse}
\attrib{Lord Byron (1788--1824)}


\section{Acknowledgments}
I would like to thank potatoes. They are great.


\newpage
\section{Abstract}

\newpage
 

\tableofcontents*{}


\newpage
\listoffigures

\listoftables

\lstlistoflistings


\mainmatter

\chapter[Introduction][Introduction]{Introduction}

\subfile{../Chapters/Introduction/Chapter_Introduction}


\chapter[Ice Theory][Ice Theory]{The theory of ice cores}

\subfile{../Chapters/IceTheory/Chapter_IceTheory}



\chapter[Data][Data]{Isotopic Data: Laki to Tambora as seen in N Ice Cores.}

\subfile{../Chapters/Data/Chapter_Data}


\chapter[Signal Analysis][Signal Analysis]{Signal Restoration and Peak Detection}

\subfile{../Chapters/SignalAnalysis/Chapter_SignalAnalysis}



\chapter[Temperature Reconstruction]{Community Firn Model and Temperature Reconstruction}

\subfile{../Chapters/TemperatureReconstruction/Chapter_TemperatureReconstruction}


\chapter[Layer Counting][Layer Counting]{Layer Counting and Annual Layer Thickness Estimation}

\subfile{../Chapters/LayerCounting/Chapter_LayerCounting}

\chapter[Method][Method]{General Method and Algorithm Walk-Throughs}

\subfile{../Chapters/Method/Chapter_Method}

\backmatter

\bibliographystyle{plain} 
\bibliography{/home/thea/Documents/Bibliographies/MasterThesis.bib}

	
\end{document}